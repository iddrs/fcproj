\documentclass[12pt, a4paper]{report}
\usepackage[T1]{fontenc}
\usepackage[utf8]{inputenc}
\usepackage[brazil]{babel}
\usepackage{graphicx}
\usepackage{float}
\usepackage[tmargin=2cm, bmargin=1.5cm, lmargin=2cm, rmargin=1.5cm]{geometry}
\usepackage{makecell}
\usepackage{booktabs}
\usepackage{longtable}
\usepackage{lscape}
\usepackage{hyperref}


\title{Fluxo de Caixa Projetado por Fonte de Recursos}
\author{
	Everton da Rosa\\
	Contador\\
	CRC RS-076595/O-3
}
\input{"cache/date.tex"}


\renewcommand{\abstractname}{Preâmbulo}
\renewcommand{\abstract}{\begin{center}
\textbf{Preâmbulo}
\end{center}}


\begin{document}

\maketitle

\begin{abstract}

Este relatório apresenta o fluxo de caixa projetado por fonte de recursos com base nos dados de~\thedate.

O objetivo é demonstrar a estimativa para o saldo de caixa final em 31 de dezembro do corrente exercício tendo por base as premissas expressas em~\nameref{ch:metodologia}.

\end{abstract}




\chapter*{Metodologia} \label{ch:metodologia}

O fluxo de caixa projetado é o demonstrado no~\nameref{app:fluxo_de_caixa} e foi elaborado a partir dos dados orçamentários e contábeis acumulados até a competência~\thedate.

As premissas utilizadas foram as seguintes:

\begin{description}
\item[Saldo atual] entende que o reflexo de toda a arrecadação que foi agregada ao saldo financeiro inicial do ano está refletida no saldo atual.

\item[A arrecadar] considera os valores estimados a arrecadar desde o mês imediatamente posterior ao de referência até o final do exercício.\\Considera o maior dos valores entre o valor já arrecadado, a previsão atualizada da receita e a soma do arrecadado acrescido do previsto a arrecadar.

\item[A empenhar] considera que a totalidade da dotação atualizada será empregada no ano. Por isso é apurado a partir da subtração do valor empenhado da dotação atualizada.

\item[Empenhado a pagar] representa os valores empenhados, liquidados e não liquidados ainda não pagos. Expressa a premissa de que o total empenhado será pago no ano.

\item[Saldo de RP] considera que a totalidade dos restos a pagar de anos anteriores será quitada dentro do exercício.

\item[Saldo extra-orçamentário] uma vez que a receita extra-orçamentária integra o Saldo atual, é importante deduzir a despesa extra-orçamentária a pagar (reduzido dos valores a compensar) do saldo financeiro.
\end{description}

\chapter*{Resultado Líquido dos Recursos Próprios} \label{ch:resultado_recursos_proprios}

Os recursos próprios englobam as fontes de recurso \textit{0001 Livre}, \textit{0020 MDE} e \textit{0040 ASPS} as quais, dada a sua natureza podem ser agrupadas para fins de análise.

Considerando a sistemática de projeção, eventualmente algumas fontes de recursos vinculados (não próprios) podem apresentar Saldo final negativo. Nesta situação, os recursos próprios devem realizar a cobertura dessa insuficiência projetada.

Nesse sentido, verifica-se que até o mês de~\thedate~as fontes de recurso vinculadas\footnote{Não é considerado o recurso vinculado 0050 RPPS} apresentam déficit projetado de R\$~\deficitVinculado. Como resultado, a fonte de recursos próprios apresenta um resultado líquido projetado de R\$~\resultadoProprio.



\vspace{16pt}

\begin{center}
	Independência, RS, \today
\end{center}

\vspace{36pt}

\begin{center}
	EVERTON DA ROSA\\
	Contador\\
	CRC RS-076595/O-3
\end{center}


\appendix

\begin{landscape}
\chapter*{Fluxo de Caixa Projetado} \label{app:fluxo_de_caixa}

\input{"cache/tabela.tex"}
\end{landscape}

\input{"cache/vinculos.tex"}


\end{document}
